\documentclass[12pt]{article}
\usepackage{amsmath}
\newcommand{\myvec}[1]{\ensuremath{\begin{pmatrix}#1\end{pmatrix}}}
\newcommand{\mydet}[1]{\ensuremath{\begin{vmatrix}#1\end{vmatrix}}}
\newcommand{\solution}{\noindent \textbf{Solution: }}
\providecommand{\brak}[1]{\ensuremath{\left(#1\right)}}
\providecommand{\norm}[1]{\left\lVert#1\right\rVert}
\let\vec\mathbf

\title{Linear Equation In Two Variables}
\author{karthik(karthik.pyla@sriprakashschools.com)}

\begin{document}
\maketitle
\section*{Class 10$^{th}$ Maths - Chapter 3}
This is Problem-1(ii) from Exercise 3.3
\item  x–y=3 , 2x–3y=36 \\

\solution \\
Given Data: x–y=3 , 2x–3y=36

This can also be written as:
\begin{align}
\\a1 \myvec{1\\2} a2 \myvec{-1\\3}b\myvec{3\\36}
\end{align}
\begin{align}
\\x = \frac{\mydet{ b & a2}}{\mydet{ a1 & a2}} =&
\frac{\mydet{ 3 & -1\\ 36 & 3}}{\mydet{1&-1\\2&3}}= \frac{\mydet{ 9 &-(-36) }}{\mydet{ 3&-(-2) }}= \frac{45}{5}=9
\end{align}
\begin{align}
\\y = \frac{\mydet{ a1 & b}}{\mydet{ a1 & a2}} =&
\frac{\mydet{ 1 & 3 \\ 2 & 36}}{\mydet{1&-1\\2&3}}=   \frac{\mydet{ 36 & -6}}{\mydet{ 3 & -(-2)}} = \frac{30}{5}=6 
\end{align}

 




\end{document}